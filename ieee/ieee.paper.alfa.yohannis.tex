\documentclass[conference]{IEEEtran}
\IEEEoverridecommandlockouts
% The preceding line is only needed to identify funding in the first footnote. If that is unneeded, please comment it out.
%Template version as of 6/27/2024

%\usepackage[colorlinks=true, linkcolor=black, citecolor=blue, urlcolor=blue]{hyperref}
\usepackage{url}
\usepackage[table]{xcolor}


\usepackage{cite}
\usepackage{amsmath,amssymb,amsfonts}
\usepackage{algorithmic}
\usepackage{graphicx}
\usepackage{textcomp}
\usepackage{xcolor}
\def\BibTeX{{\rm B\kern-.05em{\sc i\kern-.025em b}\kern-.08em
	T\kern-.1667em\lower.7ex\hbox{E}\kern-.125emX}}
\begin{document}

%\title{Conference Paper Title*\\
	%{\footnotesize \textsuperscript{*}Note: Sub-titles are not captured for https://ieeexplore.ieee.org  and
		%should not be used}
	%\thanks{Identify applicable funding agency here. If none, delete this.}
	%}
%
%\author{\IEEEauthorblockN{1\textsuperscript{st} Given Name Surname}
	%\IEEEauthorblockA{\textit{dept. name of organization (of Aff.)} \\
		%\textit{name of organization (of Aff.)}\\
		%City, Country \\
		%email address or ORCID}
	%\and
	%\IEEEauthorblockN{2\textsuperscript{nd} Given Name Surname}
	%\IEEEauthorblockA{\textit{dept. name of organization (of Aff.)} \\
		%\textit{name of organization (of Aff.)}\\
		%City, Country \\
		%email address or ORCID}
	%\and
	%\IEEEauthorblockN{3\textsuperscript{rd} Given Name Surname}
	%\IEEEauthorblockA{\textit{dept. name of organization (of Aff.)} \\
		%\textit{name of organization (of Aff.)}\\
		%City, Country \\
		%email address or ORCID}
	%\and
	%\IEEEauthorblockN{4\textsuperscript{th} Given Name Surname}
	%\IEEEauthorblockA{\textit{dept. name of organization (of Aff.)} \\
		%\textit{name of organization (of Aff.)}\\
		%City, Country \\
		%email address or ORCID}
	%\and
	%\IEEEauthorblockN{5\textsuperscript{th} Given Name Surname}
	%\IEEEauthorblockA{\textit{dept. name of organization (of Aff.)} \\
		%\textit{name of organization (of Aff.)}\\
		%City, Country \\
		%email address or ORCID}
	%\and
	%\IEEEauthorblockN{6\textsuperscript{th} Given Name Surname}
	%\IEEEauthorblockA{\textit{dept. name of organization (of Aff.)} \\
		%\textit{name of organization (of Aff.)}\\
		%City, Country \\
		%email address or ORCID}
	%}

\title{Deriving IT Strategies from Times Higher Education Ranking}


\author{\IEEEauthorblockN{1\textsuperscript{st} Alfa Yohannis\IEEEauthorrefmark{1}}
	\IEEEauthorblockA{\textit{Department of Informatics} \\
		\textit{Universitas Pradita}\\
		Tangerang, Indonesia \\
		\IEEEauthorrefmark{1}alfa.ryano@gmail.com}
	\and
	\IEEEauthorblockN{2\textsuperscript{nd} Alexander Waworuntu}
	\IEEEauthorblockA{\textit{Department of Informatics} \\
		\textit{Universitas Multimedia Nusantara}\\
		Tangerang, Indonesia\\
		alex.wawo@umn.ac.id}
	\and
	\IEEEauthorblockN{3\textsuperscript{rd} Master Edison Siregar}
	\IEEEauthorblockA{\textit{Department of Informatics} \\
		\textit{Universitas Pradita}\\
		Tangerang, Indonesia \\
		master.edison@pradita.ac.id}
	%	%\and
	%	%\IEEEauthorblockN{4\textsuperscript{th} Given Name Surname}
	%	%\IEEEauthorblockA{\textit{dept. name of organization (of Aff.)} \\
		%		%\textit{name of organization (of Aff.)}\\
		%		%City, Country \\
		%		%email address or ORCID}
	%	%\and
	%	%\IEEEauthorblockN{5\textsuperscript{th} Given Name Surname}
	%	%\IEEEauthorblockA{\textit{dept. name of organization (of Aff.)} \\
		%		%\textit{name of organization (of Aff.)}\\
		%		%City, Country \\
		%		%email address or ORCID}
	%	%\and
	%	%\IEEEauthorblockN{6\textsuperscript{th} Given Name Surname}
	%	%\IEEEauthorblockA{\textit{dept. name of organization (of Aff.)} \\
		%		%\textit{name of organization (of Aff.)}\\
		%		%City, Country \\
		%		%email address or ORCID}
	%	}

% \author{\textbf{[Hidden for double-blind review]}}
%\author{\IEEEauthorblockN{
		%		Alfa Yohannis%\IEEEauthorrefmark{1}
		%		\IEEEauthorrefmark{2},
		%		Master Edison Siregar%\IEEEauthorrefmark{1}%\IEEEauthorrefmark{3}
		%	}
	%	\IEEEauthorblockA{
		%		% \IEEEauthorrefmark{1}
		%		Hidden for Double-blind Review
		%		Department of Informatics\\
		%		Pradita University, Tangerang, Indonesia\\
		%		\IEEEauthorrefmark{2}alfa.ryano@pradita.ac.id}
	%	% 	% \IEEEauthorblockA{
		%		% 	% 	\IEEEauthorrefmark{2}Department of Computer Science\\
		%		% 	% 	University of York, York, United Kingdom}
}

\newcommand{\al}[1]{{\textbf{\color{blue} Al: #1}}}



%Add the following code before the \maketitle command:
\IEEEoverridecommandlockouts
%979-8-3315-4276-4/25/$31.00 ©2025 IEEE
\IEEEpubid{\makebox[\columnwidth]{979-8-3315-4276-4/25/\$31.00~\copyright2025 IEEE \hfill} \hspace{\columnsep}\makebox[\columnwidth]{ }}

\maketitle

%Add the following code after the \maketitle command:
\IEEEpubidadjcol
%Make sure you use the appropriate IEEE copyright notice string, as explained above.


\begin{abstract}
	This study explores the correlations among key metrics from the Times Higher Education (THE) rankings over a 10-year period to derive actionable IT strategies for universities. Using rigorous analysis, including non-parametric correlation, the research identifies the most influential variables affecting institutional performance, such as research and teaching scores, citations, and internationalisation. The findings underscore the importance of aligning IT investments with these critical areas to enhance institutional rankings and competitiveness. Based on the analysis, targeted IT strategies are recommended to support the performance of higher education institutions based on the influential variables.
\end{abstract}

\begin{IEEEkeywords}
	IT strategy, higher education, times higher education rankings, correlation analysis
\end{IEEEkeywords}


\section{Introduction}


The role of Information Technology (IT) strategy has become increasingly significant across various sectors, including Higher Education~\cite{hashim2021higher}. In an era where digital transformation drives innovation, IT strategies are crucial for achieving institutional goals, enhancing operational efficiency, and improving the quality of services provided~\cite{rahmadi2024research}. Within the context of Higher Education, these strategies are vital for ensuring that institutions remain competitive in an increasingly dynamic academic environment~\cite{fernandez2023digital}.

IT strategies in Higher Education are often developed through diverse methodologies, including strategic alignment with institutional goals, benchmarking against peer institutions, and leveraging frameworks such as enterprise architecture~\cite{bianchi2023it}. These approaches aim to optimise the utilisation of technology to support academic excellence, enhance research outputs, and improve student engagement~\cite{digitalsystems2022strategy}.

Moreover, global ranking systems, such as those provided by \textit{Times Higher Education (THE)}, introduce variables that significantly influence universities' positions in these rankings~\cite{times2023methodology}. Factors such as research output, teaching quality, industry collaboration, and international outlook play a pivotal role in determining an institution’s rank~\cite{times2022rankings}. By analysing the relationships between these variables, it becomes possible to identify which aspects should be prioritised within IT strategies to maximise their impact on institutional performance.


This study aims to explore the relationships among key variables highlighted by \textit{Times Higher Education (THE)} and how these insights can inform the development of effective IT strategies tailored to the needs of Higher Education institutions. Through an analysis of the correlations between these variables, patterns and dependencies are identified, highlighting critical areas of focus for IT investments. Based on this analysis, a set of IT strategies is proposed to specifically target improvements in the variables most significant to universities' performance.

This paper is structured into several sections. The Introduction highlights the significance of IT strategy in Higher Education, while the Related Work reviews studies on deriving such strategies, focusing on methodologies and the role of \textit{Times Higher Education} rankings. The Methodology describes the analysis of these rankings' variables. The Results and Discussion present key findings, followed by IT Strategy Recommendations offering actionable insights. Finally, the Conclusion and Future Work summarise the study's contributions and suggest future research directions.


\section{Literature Review}
\label{sec:literature_review}

The development of an Information Technology (IT) strategy aligns technology initiatives with organisational objectives.

Drechsler and Weißschädel~\cite{drechsler2018framework} introduced a framework tailored for SMEs, integrating theoretical foundations with empirical validation. Their iterative approach enables dynamic adaptation to organisational challenges. Similarly, El Alami and Belemlih~\cite{elalami2021digital} explored the integration of IT governance frameworks like COBIT for digital strategy formulation, emphasising governance as a crucial factor in aligning IT objectives with business goals. De Haes and Van Grembergen~\cite{dehaes2009governance} reinforced the importance of IT governance, highlighting maturity models as tools to assess and enhance IT alignment with strategic objectives while stressing executive involvement in decision-making.

Ensuring alignment between IT and business objectives remains a key concern. Adama et al.~\cite{adama2024alignment} examined frameworks that promote this alignment, advocating for a forward-looking approach where IT anticipates market demands and technological shifts. Complementing this perspective, Putra et al.~\cite{putra2022trends} conducted a systematic review of IT strategy implementation, identifying common challenges such as resource constraints, resistance to change, and difficulties in integrating legacy systems with emerging technologies.

The evolving landscape of digital transformation further complicates IT strategy formulation. Rahmadi~\cite{rahmadi2024research} recognised digital transformation as a pivotal driver but noted that many organisations struggle to keep pace with technological advancements, necessitating adaptive strategies. In higher education, Fern\'andez et al.~\cite{fernandez2023digital} examined digital transformation initiatives, underscoring the role of stakeholder collaboration in shaping effective IT strategies.

Amidst increasing IT complexity, new methodologies and technologies are being leveraged for strategic planning. Machine learning and predictive analytics have emerged as valuable tools for forecasting IT investment impacts, facilitating data-driven decision-making, and enhancing organisational agility~\cite{digitalsystems2022strategy}.

Another significant factor influencing IT strategies is the use of performance metrics, particularly in higher education. Institutions rely on global rankings such as Times Higher Education (THE), which prioritise factors like research output, teaching quality, and international collaboration. Aligning IT strategies with these indicators helps institutions strengthen their global standing~\cite{times2023methodology}. Da Silva and Santos~\cite{dasilva2014cobit} demonstrated that ranking metrics serve as benchmarks for IT investment prioritisation, ensuring that technological initiatives contribute directly to institutional performance. Similarly, Uslu~\cite{uslu2020university} analysed key ranking indicators, illustrating how research, teaching, and collaboration inform strategic decisions aimed at improving competitiveness.

A complementary perspective is offered by studies on university competitiveness through data analytics. A predictive model proposed by~\cite{analytics2022competitiveness} leverages historical ranking data and institutional metrics to identify areas requiring improvement. These models enable universities to refine their strategies and ensure that IT investments support metrics with the highest impact. While the direct connection between THE metrics and IT strategy development is not always explicit, the collective findings from these studies suggest practical insights. By strategically leveraging IT, institutions can enhance research, teaching, and global collaboration—critical drivers of improved ranking outcomes. This study extends these insights by examining THE metrics over a decade, identifying key variables that shape IT strategies and optimising institutional objectives accordingly.





\section{Methodology}
\label{sec:methodology}

The research was performed quantitatively, with data collected through web scrapping. Python scripts were written to process, analyse, and visualise the collected data using its various libraries. 

\subsection{Data Collection and Cleaning}

The data scrapping was achieved by collecting universities' ranks and rank factors by scrapping scores from 2013 to 2023 in the Times Higher Education (THE) website~\cite{the2024} and combining it with the data of university ranks obtained from Kaggle~\cite{ONeil_2020}\footnote{The contribution of this dataset was minimal, mainly for years 2011-2012.}, producing more than 11-year data span with 711 universities involved. 

After collecting the necessary data, it underwent several data-cleaning steps to ensure it was ready for analysis. Empty numeric values were filled using linear estimation, logarithmic estimation, etc., based on the coefficient of determination $R^{2}$ to estimate the missing values. 
%The filled values were then normalised and added as a new columns to the dataset. 
%Secondly, empty non-numeric values were filled with `n/a' to help handle missing non-numeric values in the following steps.

Moreover, unnecessary columns were removed to streamline the dataset, and only several columns were used in the analysis as variables. These variables were overall score, rank, research score, teaching score, citations score, intl students, international outlook score, student staff ratio, industry income score, number students, and year. 

\subsection{Data Analysis}

Correlations were calculated between the variables\footnote{Code and data can be found at:
	[hidden for double-blind review] 
	%	\url{https://github.com/alfa-yohannis/the-metrics-and-it-strategy/tree/main/code}
}. 
The analysis assessed the strength of the relationships using correlation coefficients, along with their statistical significance. Spearman's Rank Correlation Coefficient~\cite{spearman1904general} was employed to ensure the robustness of the results, as it does not assume normality in the data distribution. The significance of the correlations was evaluated to determine whether the observed relationships were likely to have occurred by chance. 


\begin{table*}[h!]
	\centering
	\caption{Consolidated Correlation Analysis Results ordered by the Influence column}
	
	\label{tab:correlation_combined}
	\scriptsize
	\begin{tabular}{|l|r|r|r|r|r|r|r|r|r|r|r|r|}
		\hline
		\textbf{Variable} & \textbf{OS} & \textbf{R} & \textbf{RS} & \textbf{TS} & \textbf{CS} & \textbf{IS} & \textbf{IO} & \textbf{II} & \textbf{SSR} & \textbf{NS} & \textbf{Y} & \textbf{Influence} \\
		\hline
		
		Overall Score & \cellcolor{gray!50}1.00 & \cellcolor{gray!44}-0.89\textsuperscript{***} & \cellcolor{gray!42}0.86\textsuperscript{***} & \cellcolor{gray!40}0.81\textsuperscript{***} & \cellcolor{gray!35}0.72\textsuperscript{***} & \cellcolor{gray!19}0.39\textsuperscript{***} & \cellcolor{gray!16}0.32\textsuperscript{***} & \cellcolor{gray!17}0.34\textsuperscript{***} & \cellcolor{gray!12}-0.24\textsuperscript{***} & \cellcolor{gray!07}0.14\textsuperscript{***} & \cellcolor{gray!09}0.19\textsuperscript{***} & 5.90 \\
		
		Rank & \cellcolor{gray!44}-0.89\textsuperscript{***} & \cellcolor{gray!50}1.00 & \cellcolor{gray!42}-0.84\textsuperscript{***} & \cellcolor{gray!41}-0.82\textsuperscript{***} & \cellcolor{gray!35}-0.70\textsuperscript{***} & \cellcolor{gray!19}-0.38\textsuperscript{***} & \cellcolor{gray!13}-0.27\textsuperscript{***} & \cellcolor{gray!14}-0.29\textsuperscript{***} & \cellcolor{gray!12}0.24\textsuperscript{***} & \cellcolor{gray!08}-0.15\textsuperscript{***} & \cellcolor{gray!08}0.15\textsuperscript{***} & 5.73 \\
		
		Research Score & \cellcolor{gray!42}0.86\textsuperscript{***} & \cellcolor{gray!42}-0.84\textsuperscript{***} & \cellcolor{gray!50}1.00 & \cellcolor{gray!42}0.86\textsuperscript{***} & \cellcolor{gray!22}0.45\textsuperscript{***} & \cellcolor{gray!16}0.31\textsuperscript{***} & \cellcolor{gray!11}0.22\textsuperscript{***} & \cellcolor{gray!22}0.45\textsuperscript{***} & \cellcolor{gray!08}-0.16\textsuperscript{***} & \cellcolor{gray!10}0.20\textsuperscript{***} & \cellcolor{gray!04}0.08 & 5.35 \\
		
		Teaching Score & \cellcolor{gray!40}0.81\textsuperscript{***} & \cellcolor{gray!41}-0.82\textsuperscript{***} & \cellcolor{gray!42}0.86\textsuperscript{***} & \cellcolor{gray!50}1.00 & \cellcolor{gray!21}0.43\textsuperscript{***} & \cellcolor{gray!10}0.20\textsuperscript{***} & \cellcolor{gray!02}0.03\textsuperscript{*} & \cellcolor{gray!19}0.38\textsuperscript{***} & \cellcolor{gray!17}-0.34\textsuperscript{***} & \cellcolor{gray!10}0.20\textsuperscript{***} & \cellcolor{gray!01}0.02 & 5.07 \\
		
		Citations Score & \cellcolor{gray!35}0.72\textsuperscript{***} & \cellcolor{gray!35}-0.70\textsuperscript{***} & \cellcolor{gray!22}0.45\textsuperscript{***} & \cellcolor{gray!21}0.43\textsuperscript{***} & \cellcolor{gray!50}1.00 & \cellcolor{gray!16}0.32\textsuperscript{***} & \cellcolor{gray!15}0.30\textsuperscript{***} & \cellcolor{gray!03}0.05\textsuperscript{***} & \cellcolor{gray!11}-0.23\textsuperscript{***} & \cellcolor{gray!01}0.02 & \cellcolor{gray!09}0.18\textsuperscript{***} & 4.40 \\
		
		International Students & \cellcolor{gray!19}0.39\textsuperscript{***} & \cellcolor{gray!19}-0.38\textsuperscript{***} & \cellcolor{gray!16}0.31\textsuperscript{***} & \cellcolor{gray!10}0.20\textsuperscript{***} & \cellcolor{gray!16}0.32\textsuperscript{***} & \cellcolor{gray!50}1.00 & \cellcolor{gray!41}0.82\textsuperscript{***} & \cellcolor{gray!01}0.02 & \cellcolor{gray!04}-0.07\textsuperscript{***} & \cellcolor{gray!09}-0.18\textsuperscript{***} & \cellcolor{gray!05}0.09\textsuperscript{***} & 3.76 \\
		
		International Outlook & \cellcolor{gray!16}0.32\textsuperscript{***} & \cellcolor{gray!13}-0.27\textsuperscript{***} & \cellcolor{gray!11}0.22\textsuperscript{***} & \cellcolor{gray!02}0.03\textsuperscript{*} & \cellcolor{gray!15}0.30\textsuperscript{***} & \cellcolor{gray!41}0.82\textsuperscript{***} & \cellcolor{gray!50}1.00 & \cellcolor{gray!01}0.01 & \cellcolor{gray!06}0.11\textsuperscript{***} & \cellcolor{gray!08}-0.15\textsuperscript{***} & \cellcolor{gray!12}0.24\textsuperscript{***} & 3.47 \\
		
		Industry Income & \cellcolor{gray!17}0.34\textsuperscript{***} & \cellcolor{gray!14}-0.29\textsuperscript{***} & \cellcolor{gray!22}0.45\textsuperscript{***} & \cellcolor{gray!19}0.38\textsuperscript{***} & \cellcolor{gray!03}0.05\textsuperscript{***} & \cellcolor{gray!01}0.02 & \cellcolor{gray!01}0.01 & \cellcolor{gray!50}1.00 & \cellcolor{gray!00}0.00 & \cellcolor{gray!02}0.03\textsuperscript{**} & \cellcolor{gray!08}0.16\textsuperscript{***} & 2.72 \\
		
		Student-to-Staff Ratio & \cellcolor{gray!12}-0.24\textsuperscript{***} & \cellcolor{gray!12}0.24\textsuperscript{***} & \cellcolor{gray!08}-0.16\textsuperscript{***} & \cellcolor{gray!17}-0.34\textsuperscript{***} & \cellcolor{gray!11}-0.23\textsuperscript{***} & \cellcolor{gray!04}-0.07\textsuperscript{***} & \cellcolor{gray!06}0.11\textsuperscript{***} & \cellcolor{gray!00}0.00 & \cellcolor{gray!50}1.00 & \cellcolor{gray!16}0.31\textsuperscript{***} & \cellcolor{gray!01}0.02 & 2.72 \\
		
		Number of Students & \cellcolor{gray!07}0.14\textsuperscript{***} & \cellcolor{gray!08}-0.15\textsuperscript{***} & \cellcolor{gray!10}0.20\textsuperscript{***} & \cellcolor{gray!10}0.20\textsuperscript{***} & \cellcolor{gray!01}0.02 & \cellcolor{gray!09}-0.18\textsuperscript{***} & \cellcolor{gray!08}-0.15\textsuperscript{***} & \cellcolor{gray!02}0.03\textsuperscript{**} & \cellcolor{gray!16}0.31\textsuperscript{***} & \cellcolor{gray!50}1.00 & \cellcolor{gray!01}0.02 & 2.38 \\
		
		Year & \cellcolor{gray!09}0.19\textsuperscript{***} & \cellcolor{gray!08}0.15\textsuperscript{***} & \cellcolor{gray!04}0.08 & \cellcolor{gray!01}0.02 & \cellcolor{gray!09}0.18\textsuperscript{***} & \cellcolor{gray!05}0.09\textsuperscript{***} & \cellcolor{gray!12}0.24\textsuperscript{***} & \cellcolor{gray!08}0.16\textsuperscript{***} & \cellcolor{gray!01}0.02 & \cellcolor{gray!01}0.02 & \cellcolor{gray!50}1.00 & 2.07 \\
		
		
		
		\hline
	\end{tabular}
	\\
	\vspace{1mm} \scriptsize \raggedright OS = Overall Score, R = Rank, RS = Research Score, TS = Teaching Score, CS = Citations Score, IS = International Students, IO = International Outlook, II = Industry Income, SSR = Student-to-Staff Ratio, NS = Number of Students, Y = Year. The \textit{Influence} is the sum of the absolute, significant correlation values.
\end{table*}


\begin{table*}[h!]
	\centering
	\caption{Consolidated Correlation Analysis Results for the X University in China ordered by the Influence column}
	\label{tab:correlation_a_university}
	\scriptsize
	\begin{tabular}{|l|r|r|r|r|r|r|r|r|r|r|r|r|}
		\hline
		\textbf{Variable} & \textbf{RS} & \textbf{Y} & \textbf{R} & \textbf{NS} & \textbf{CS} & \textbf{SSR} & \textbf{OS} & \textbf{TS} & \textbf{IO} & \textbf{II} & \textbf{IS} & \textbf{Influence} \\
		\hline
		
		Research Score & \cellcolor{gray!49}1.00 & \cellcolor{gray!49}0.99\textsuperscript{***} & \cellcolor{gray!36}-0.72\textsuperscript{**} & \cellcolor{gray!47}0.95\textsuperscript{***} & \cellcolor{gray!41}0.82\textsuperscript{***} & \cellcolor{gray!39}0.78\textsuperscript{**} & \cellcolor{gray!39}0.79\textsuperscript{**} & \cellcolor{gray!32}0.65\textsuperscript{**} & \cellcolor{gray!28}0.57\textsuperscript{*} & \cellcolor{gray!10}-0.21 & \cellcolor{gray!11}-0.23 & 11.40 \\
				
		Year & \cellcolor{gray!49}0.99\textsuperscript{***} & \cellcolor{gray!50}1.00\textsuperscript{***} & \cellcolor{gray!34}-0.69\textsuperscript{**} & \cellcolor{gray!49}0.98\textsuperscript{***} & \cellcolor{gray!42}0.84\textsuperscript{***} & \cellcolor{gray!42}0.85\textsuperscript{***} & \cellcolor{gray!40}0.81\textsuperscript{***} & \cellcolor{gray!31}0.62\textsuperscript{*} & \cellcolor{gray!28}0.57\textsuperscript{*} & \cellcolor{gray!12}-0.25\textsuperscript{} & \cellcolor{gray!15}-0.31\textsuperscript{} & 10.30 \\
	
		Rank & \cellcolor{gray!36}-0.72\textsuperscript{**} & \cellcolor{gray!34}-0.69\textsuperscript{**} & \cellcolor{gray!50}1.00\textsuperscript{***} & \cellcolor{gray!34}-0.68\textsuperscript{**} & \cellcolor{gray!44}-0.88\textsuperscript{***} & \cellcolor{gray!34}-0.68\textsuperscript{**} & \cellcolor{gray!40}-0.80\textsuperscript{***} & \cellcolor{gray!8}-0.17 & \cellcolor{gray!11}-0.22 & \cellcolor{gray!31}0.63\textsuperscript{**} & \cellcolor{gray!8}0.16 & 10.12 \\
		
		Number of Students & \cellcolor{gray!47}0.95\textsuperscript{***} & \cellcolor{gray!49}0.98\textsuperscript{***} & \cellcolor{gray!34}-0.68\textsuperscript{**} & \cellcolor{gray!50}1.00\textsuperscript{***} & \cellcolor{gray!42}0.85\textsuperscript{***} & \cellcolor{gray!43}0.86\textsuperscript{***} & \cellcolor{gray!37}0.74\textsuperscript{**} & \cellcolor{gray!30}0.60\textsuperscript{*} & \cellcolor{gray!25}0.50\textsuperscript{} & \cellcolor{gray!17}-0.35\textsuperscript{} & \cellcolor{gray!13}-0.39\textsuperscript{} & 10.12 \\
		
		Citation Score & \cellcolor{gray!44}0.82\textsuperscript{***} & \cellcolor{gray!43}0.84\textsuperscript{***} & \cellcolor{gray!50}-0.88\textsuperscript{***} & \cellcolor{gray!44}0.85\textsuperscript{***} & \cellcolor{gray!50}1.00\textsuperscript{***} & \cellcolor{gray!44}0.85\textsuperscript{***} & \cellcolor{gray!44}0.82\textsuperscript{***} & \cellcolor{gray!09}0.36\textsuperscript{} & \cellcolor{gray!19}0.39\textsuperscript{} & \cellcolor{gray!20}-0.52\textsuperscript{} & \cellcolor{gray!15}-0.31\textsuperscript{} & 10.12 \\
		
		Student-to-Staff Ratio & \cellcolor{gray!36}0.78\textsuperscript{***} & \cellcolor{gray!44}0.85\textsuperscript{***} & \cellcolor{gray!22}-0.68\textsuperscript{**} & \cellcolor{gray!44}0.86\textsuperscript{***} & \cellcolor{gray!44}0.85\textsuperscript{***} & \cellcolor{gray!50}1.00\textsuperscript{***} & \cellcolor{gray!44}0.85\textsuperscript{***} & \cellcolor{gray!11}0.28\textsuperscript{} & \cellcolor{gray!21}0.43\textsuperscript{} & \cellcolor{gray!20}-0.50\textsuperscript{} & \cellcolor{gray!18}-0.46\textsuperscript{} & 9.76 \\
		
		Overall Score & \cellcolor{gray!39}0.79\textsuperscript{**} & \cellcolor{gray!40}0.81\textsuperscript{***} & \cellcolor{gray!40}-0.80\textsuperscript{***} & \cellcolor{gray!37}0.74\textsuperscript{**} & \cellcolor{gray!41}0.82\textsuperscript{***} & \cellcolor{gray!42}0.85\textsuperscript{***} & \cellcolor{gray!50}1.00\textsuperscript{***} & \cellcolor{gray!13}0.27 & \cellcolor{gray!19}0.39 & \cellcolor{gray!23}-0.47\textsuperscript{*} & \cellcolor{gray!11}-0.23 & 9.62 \\
		
		Teaching Score & \cellcolor{gray!32}0.65\textsuperscript{**} & \cellcolor{gray!31}0.62\textsuperscript{**} & \cellcolor{gray!8}-0.17 & \cellcolor{gray!30}0.60\textsuperscript{**} & \cellcolor{gray!18}0.36 & \cellcolor{gray!14}0.28 & \cellcolor{gray!13}0.27 & \cellcolor{gray!50}1.00\textsuperscript{***} & \cellcolor{gray!37}0.74\textsuperscript{**} & \cellcolor{gray!8}0.16 & \cellcolor{gray!11}0.23 & 2.78 \\
		
		International Outlook & \cellcolor{gray!28}0.57\textsuperscript{*} & \cellcolor{gray!28}0.57\textsuperscript{*} & \cellcolor{gray!11}-0.22\textsuperscript{} & \cellcolor{gray!25}0.50\textsuperscript{} & \cellcolor{gray!19}0.39\textsuperscript{} & \cellcolor{gray!21}0.43\textsuperscript{} & \cellcolor{gray!19}0.39\textsuperscript{} & \cellcolor{gray!37}0.74\textsuperscript{**} & \cellcolor{gray!50}1.00\textsuperscript{***} & \cellcolor{gray!13}0.26\textsuperscript{} & \cellcolor{gray!23}0.47\textsuperscript{} & 1.48 \\
		
		Industry Income & \cellcolor{gray!10}-0.21\textsuperscript{} & \cellcolor{gray!12}-0.25\textsuperscript{} & \cellcolor{gray!31}0.63\textsuperscript{**} & \cellcolor{gray!17}-0.35\textsuperscript{} & \cellcolor{gray!26}-0.52\textsuperscript{} & \cellcolor{gray!25}-0.50\textsuperscript{} & \cellcolor{gray!23}-0.47\textsuperscript{} & \cellcolor{gray!8}0.16\textsuperscript{} & \cellcolor{gray!13}0.26\textsuperscript{} & \cellcolor{gray!50}1.00\textsuperscript{***} & \cellcolor{gray!19}0.39\textsuperscript{} & 1.26 \\
		
		International Students & \cellcolor{gray!11}-0.23\textsuperscript{} & \cellcolor{gray!15}-0.31\textsuperscript{} & \cellcolor{gray!8}0.16\textsuperscript{} & \cellcolor{gray!19}-0.39\textsuperscript{} & \cellcolor{gray!15}-0.31\textsuperscript{} & \cellcolor{gray!23}-0.46\textsuperscript{} & \cellcolor{gray!11}-0.23\textsuperscript{} & \cellcolor{gray!11}0.23\textsuperscript{} & \cellcolor{gray!23}0.47\textsuperscript{} & \cellcolor{gray!19}0.39\textsuperscript{} & \cellcolor{gray!50}1.00\textsuperscript{***} & 1.00 \\
		
		\hline
	\end{tabular}
\end{table*}




\section{Results and Discussion}
This section examines the correlations among THE variables, including international outlook score, rank, citations score, overall score, industry income score, international students, research score, number of students, student-to-staff ratio, and teaching score. 

\textbf{Overall Score}. The correlation analysis highlights key factors associated with the \textit{overall score}, emphasizing its role as a comprehensive measure of institutional performance (Table \ref{tab:correlation_combined}). Among the variables examined, the \textit{rank} ($r = -0.89$), \textit{research score} ($r = 0.86$), and \textit{teaching score} ($r = 0.81$) exhibit the strongest correlations. The strong negative correlation with \textit{rank} underscores the critical relationship between overall institutional performance and competitive standings. Similarly, the high positive correlations with \textit{research score} and \textit{teaching score} highlight the importance of academic and research excellence in achieving higher overall scores.

Moderate correlations are observed for \textit{citations score} ($r = 0.72$), \textit{international students} ($r = 0.39$), \textit{industry income score} ($r = 0.34$), and \textit{international outlook score} ($r = 0.32$). These findings suggest that impactful research, global engagement, and industry collaborations substantially contribute to shaping the overall score.


\textbf{Rank}. The correlation analysis highlights key determinants influencing institutional rankings (Table \ref{tab:correlation_combined}). Among the examined variables, the \textit{overall score}, \textit{research score}, \textit{teaching score}, and \textit{citations score} exhibit the strongest negative correlations, underscoring their significant impact on rank. The \textit{overall score} ($r = -0.89$) and \textit{research score} ($r = -0.84$) emerge as the most influential factors, emphasizing their critical roles in driving institutional performance and competitiveness.

Moderate negative correlations are observed for \textit{teaching score} ($r = -0.82$), \textit{citations score} ($r = -0.70$), and \textit{international students} ($r = -0.38$), highlighting the importance of academic quality, research impact, and global engagement in shaping rankings. Weaker negative correlations are found for \textit{industry income score} ($r = -0.29$) and \textit{international outlook score} ($r = -0.27$), reflecting their secondary influence on institutional rankings.


\textbf{Research Score}. The correlation analysis highlights critical factors contributing to the \textit{research score}, emphasizing its central role in institutional performance (Table \ref{tab:correlation_combined}). Among the examined variables, the \textit{overall score} ($r = 0.86$) and \textit{teaching score} ($r = 0.86$) exhibit the strongest positive correlations, underscoring their close alignment with research activities. Enhancing both teaching and overall institutional quality directly supports improvements in research performance.

The \textit{rank} ($r = -0.84$) demonstrates a strong negative correlation, reaffirming the inverse relationship where higher research scores correspond to better rankings (lower numerical rank values). This finding underscores the pivotal role of research excellence in achieving competitive standings.

Moderate correlations are observed for \textit{industry income score} ($r = 0.45$), \textit{citations score} ($r = 0.45$), and \textit{international students} ($r = 0.31$). These results suggest that impactful research outputs, industry partnerships, and global engagement significantly influence research scores. The \textit{international outlook score} ($r = 0.22$) also reflects the contribution of internationalization, though to a lesser extent.


\textbf{Teaching Score}. The correlation analysis highlights key factors contributing to the \textit{teaching score}, emphasizing its critical role in institutional performance (Table \ref{tab:correlation_combined}). Among the examined variables, the \textit{research score} ($r = 0.86$), \textit{rank} ($r = -0.82$), and \textit{overall score} ($r = 0.81$) exhibit the strongest correlations. The positive correlations with \textit{research score} and \textit{overall score} underscore the alignment of teaching excellence with broader institutional quality and academic outputs. Conversely, the strong negative correlation with \textit{rank} highlights the inverse relationship, where higher teaching scores correspond to better rankings (lower numerical rank values).

Moderate correlations are observed for \textit{citations score} ($r = 0.43$), \textit{industry income score} ($r = 0.38$), and \textit{student-to-staff ratio} ($r = -0.34$). These findings indicate that impactful research, robust industry collaborations, and favorable staff-to-student ratios significantly influence teaching quality.



\textbf{Citations Score}. The correlation analysis identifies key factors associated with the \textit{citations score}, highlighting its significance as a measure of research impact (Table \ref{tab:correlation_combined}). Among the variables examined, the \textit{overall score} ($r = 0.72$) exhibits the strongest positive correlation, underscoring the alignment between high overall institutional performance and research impact through citations. The \textit{rank} ($r = -0.70$) shows a strong negative correlation, emphasizing the critical role of citations in improving institutional rankings (lower numerical rank values).

Moderate positive correlations are observed for \textit{research score} ($r = 0.45$), \textit{teaching score} ($r = 0.43$), \textit{international students} ($r = 0.32$), and \textit{international outlook score} ($r = 0.30$). These findings indicate that impactful research, quality teaching, and global engagement significantly contribute to citation performance.


\textbf{International Students}. The correlation analysis highlights key factors associated with the \textit{international students} variable, emphasizing its role in reflecting global engagement and diversity (Table \ref{tab:correlation_combined}). Among the examined variables, the \textit{international outlook score} ($r = 0.82$) exhibits the strongest positive correlation, underscoring the alignment between a diverse student body and an institution's international profile.

Moderate positive correlations are observed for \textit{overall score} ($r = 0.39$), \textit{citations score} ($r = 0.32$), \textit{research score} ($r = 0.31$), and \textit{teaching score} ($r = 0.20$). These findings indicate that institutions with a higher proportion of international students tend to excel in broader institutional performance metrics and research impact, contributing significantly to their global competitiveness.

Negative correlations are observed for \textit{rank} ($r = -0.38$) and \textit{number of students} ($r = -0.18$), suggesting that a higher percentage of international students is associated with better rankings (lower numerical rank values) and smaller overall student populations. 


\textbf{International Outlook Score}. The correlation analysis examines factors influencing the \textit{international outlook score} (Table \ref{tab:correlation_combined}), emphasizing its role in global engagement and diversity. The \textit{international students} percentage ($r = 0.82$) exhibits the strongest positive correlation, highlighting the strong link between student diversity and an institution’s international standing.

Moderate correlations exist for \textit{overall score} ($r = 0.32$), \textit{citations score} ($r = 0.30$), and \textit{rank} ($r = -0.27$), suggesting that institutions with strong performance, research impact, and better rankings tend to have higher international engagement. The \textit{year} ($r = 0.24$) also shows a moderate correlation, indicating a growing focus on internationalization over time.

\textbf{Student-to-Staff Ratio}. The correlation analysis identifies factors linked to the \textit{student-to-staff ratio} (Table \ref{tab:correlation_combined}), reflecting its role in educational quality and resource availability. The \textit{number of students} ($r = 0.31$) has the strongest positive correlation, indicating that larger institutions tend to have higher student-to-staff ratios.

The \textit{teaching score} ($r = -0.34$) shows the strongest negative correlation, suggesting that lower student-to-staff ratios contribute to better teaching outcomes. Moderate negative correlations with \textit{overall score} ($r = -0.24$), \textit{citations score} ($r = -0.23$), and \textit{research score} ($r = -0.16$) imply that lower ratios align with stronger institutional and research performance. A moderate positive correlation with \textit{rank} ($r = 0.24$) suggests that higher student-to-staff ratios correspond to poorer rankings.


%----
\textbf{Industry Income Score}. The correlation analysis identifies key factors linked to the \textit{industry income score} (Table \ref{tab:correlation_combined}), emphasizing its role in institutional partnerships and financial sustainability. The \textit{research score} ($r = 0.45$) and \textit{teaching score} ($r = 0.38$) exhibit the strongest positive correlations, highlighting the impact of strong research and teaching activities in attracting industry collaborations and funding.

The \textit{overall score} ($r = 0.34$) also shows a moderate positive correlation, reinforcing the link between institutional performance and industry income. The \textit{rank} ($r = -0.29$) displays a negative correlation, suggesting that institutions with higher industry income tend to achieve better rankings.


\textbf{Number of Students}. The correlation analysis explores factors associated with the \textit{number of students} (Table \ref{tab:correlation_combined}), emphasizing its impact on institutional structure. The \textit{student-to-staff ratio} ($r = 0.31$) exhibits the strongest positive correlation, indicating that larger institutions tend to have higher student-to-staff ratios.

Moderate correlations exist for \textit{research score} ($r = 0.20$) and \textit{teaching score} ($r = 0.20$), suggesting that larger student bodies may modestly enhance teaching and research activities. Negative correlations with \textit{international students} ($r = -0.18$), \textit{international outlook score} ($r = -0.15$), and \textit{rank} ($r = -0.15$) indicate that institutions with larger student populations may have lower international engagement and slightly weaker rankings.



\textbf{Year}. The correlation analysis examines factors linked to the \textit{year} (Table \ref{tab:correlation_combined}), revealing temporal trends in institutional metrics. The \textit{international outlook score} ($r = 0.24$) shows the strongest positive correlation, indicating a growing emphasis on global engagement.

Moderate correlations are observed for \textit{overall score} ($r = 0.19$), \textit{citations score} ($r = 0.18$), and \textit{industry income score} ($r = 0.16$). These results suggest gradual improvements in institutional performance, research impact, and financial sustainability over time.

\textbf{Total Influence}. The analysis of total influence ranks key factors in determining outcomes (Table \ref{tab:correlation_combined}). The \textit{overall score} has the highest influence (5.90), followed by \textit{rank} (5.73) and \textit{research score} (5.35), emphasizing performance-driven metrics in institutional success.

The \textit{teaching score} (5.07) and \textit{citations score} (4.40) also hold significant influence, reinforcing the importance of academic and research excellence. Internationalization-related variables, including \textit{international students} (3.76) and \textit{international outlook score} (3.47), play a moderate but supportive role in institutional outcomes.

\textbf{Institutional Case Example}. Table~\ref{tab:correlation_a_university} presents a consolidated correlation analysis for an \textsf{X} University in China (name classified), offering a snapshot of how academic and operational metrics co-vary within a single institution. Such patterns may differ widely across universities depending on their strategic orientation. Notably, \textbf{Research Score (RS)}, \textbf{Year (Y)}, and \textbf{Rank (R)} show strong intercorrelation ($|r| > 0.68$, $p < 0.01$), suggesting this is a high-performing, research-driven university. Positive correlations of RS with \textbf{Citations Score (CS)}, \textbf{Number of Students (NS)}, and \textbf{Overall Score (OS)} indicate that improvements in research coincide with wider impact and growth. Since Overall Score and Rank also correlate strongly with Year, this suggests that both institutional ranking and overall performance have improved year by year. In contrast, \textbf{Teaching Score (TS)}, \textbf{International Outlook (IO)}, and \textbf{Industry Income (II)} show weaker links to core metrics, indicating limited influence and roles.






\section{IT Strategy Recommendation}

The following strategies are based on key correlations among THE ranking indicators in Table \ref{tab:correlation_combined}. Each IT initiative targets specific performance dimensions.

\begin{enumerate}
	\item \textbf{Teaching ($r_{TS-OS}=0.81$, $r_{TS-RS}=0.86$).} Deploy a digital-first \emph{Learning Experience Platform} integrating adaptive content, virtual labs, and real-time analytics to enhance pedagogical quality and student engagement, strengthening both Teaching and Research pillars~\cite{azough2021digital,cheng2022ai}.
	
	\item \textbf{Research ($r_{RS-OS}=0.86$).} Build a \emph{Research Information Management System} (RIMS) with high-performance computing and FAIR data repositories to streamline grants, projects, and open-access dissemination—key to THE’s “Volume, Income and Reputation”~\cite{tenopir2011data,robinson2019rims}.
	
	\item \textbf{Citations ($r_{CS-OS}=0.72$).} Embed \emph{Bibliometric Dashboards} in RIMS to highlight high-impact journals, foster interdisciplinary co-authorship, and automate pre-print deposits, boosting citation rates~\cite{moed2005citation,borner2010structure}.
	
	\item \textbf{International Outlook ($r_{IS-OS}=0.82$, $r_{IO-OS}=0.39$).} Launch a multilingual \emph{Global Engagement Portal} for joint degrees, remote onboarding, and AI-supported admissions to lift international staff, student, and co-authorship metrics~\cite{boeren2023global,dewit2020internationalisation}.
	
	\item \textbf{Industry Income ($r_{II-RS}=0.45$).} Integrate a \emph{Partnership Management System} to track patents, consultancy, and contracts, unlocking commercialisation opportunities and boosting external funding~\cite{perkmann2013academic,sengupta2023value}.
	
	\item \textbf{Student–Staff Ratio ($r_{NS-SSR}=0.31$).} Extend University Enterprise Resource Planning (ERP) with \emph{Predictive Workforce \& Timetabling Analytics} to forecast staffing and optimise class sizes, improving resource efficiency~\cite{abbas2020forecasting,panagiotopoulos2021learning}.
	
	\item \textbf{Rank ($r_{OS-R}=-0.89$).} A central \emph{Strategic KPI Dashboard} should oversee all IT solutions, aligning investments with indicator weights and tracking annual progress~\cite{agbo2023dashboards,gunawardena2021digital}.
	
	\item \textbf{Year ($r_{Y-OS}=0.19$).} Maintain a \emph{A Long Term Digital Transformation Road-Map} with agile governance to stay responsive to ranking changes and technology shifts~\cite{hashim2021higher,kuzu2020digital}.
	
	\item \textbf{International Student Mix ($r_{NS-IO}=-0.18$).} Use \emph{AI-based Segmentation} and a \emph{Digital Platform} to personalise support and increase reaching international students~\cite{komljenovic2020platformisation,perkins2021ai}.
	
	\item \textbf{Data-Driven Management (cross-indicator insight).} Deploy a \emph{Unified Analytics} combining teaching, research, admissions, and finance data; advanced models (e.g.\ machine learning models) then guide measurable improvements~\cite{dede2021data,singh2020predictive}.
\end{enumerate}

\noindent
While these strategies rest on robust correlations and literature, implementation must fit each institution’s context. Factors such as maturity, infrastructure, policy, and budget affect feasibility and impact. A phased, modular approach -- supported by agile governance~\cite{wirsing2021agile} -- helps maintain relevance and generalisability across diverse university environments.



\section{Limitations and Threats to Validity}

The primary limitation of this study lies in the absence of experimental control or temporal precedence, which restricts the ability to draw robust causal inferences. While statistically significant correlations provide meaningful insights, they must be interpreted with caution due to the potential presence of confounding variables and the inherent complexity of the higher education environment. These associations should be regarded as informative rather than definitive. Furthermore, the correlation-based approach does not imply causation. Although some patterns may suggest directional tendencies, such interpretations require further verification. These limitations underscore the importance of future research employing experimental or longitudinal designs to better understand the underlying mechanisms driving the observed relationships and to enhance the credibility and robustness of the findings.


\section{Conclusions}

The analysis of correlations among institutional performance metrics highlights the key roles of research, teaching, and global engagement in university outcomes. Findings underscore the importance of metrics such as overall performance, citations, and industry collaborations in shaping success and rankings. IT strategies have been recommended to optimise these areas, including enhancing research and teaching, improving internationalisation, and leveraging industry partnerships. Future work could explore temporal trends and include data from a broader range of institutions to enhance applicability across diverse contexts.

\section*{Acknowledgment}
AI tools assisted in script development, data analysis, and language refinement. The paper’s structure was drafted by the authors and finalised for clarity and alignment with research goals. This publication is supported by Pradita University through the International Conference Grant Scheme.


\bibliography{references}
\bibliographystyle{IEEEtran}

\end{document}
